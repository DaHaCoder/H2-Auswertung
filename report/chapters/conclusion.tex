\chapter{Zusammenfassung und Ausblick}
\label{chap:conclusion}

\noindent Zusammenfassend ist zu sagen, dass es uns gelungen ist, die gesamte Hyperfeinstruktur der $D2$-Linie aufzulösen und auszuwerten. Unsere Ergebnisse entsprechen mit einigen Abweichungen den im Rahmen der Fehleranalyse diskutierten Literaturwerten. Im Allgemeinen können die Fehler durch präzisere Justage und Einkopplung des Strahls in den Referenzresonator und den polarisierenden Strahlteiler reduziert werden. Ebenso trägt die Wahl geeigneter Fit-Parameter maßgeblich dazu bei, mit welcher Genauigkeit die Peaks und damit unsere Frequenzabstände bestimmt werden. 

\noindent In unserem Experiment haben wir für eine dopplerfreie Auflösung der Hyperfeinstruktur am Rubidium die Sättigungsspektroskopie verwendet. Allerdings gibt es auch noch weitere spektroskopische Methoden wie die Mehrphotonen- und die Polarisationsspektroskopie mittels derer dopplerfreie Untersuchungen durchfgeführt werden können \cite{Demt}. Die Sättigungsspekroskopie in Gaszellen ist die simpelste dieser Formen und dennoch absolut ausreichend für die Anwendung in diesem Experiment. Mittels dieser Methode können auch weitere gasförmige Proben wie Cäsium \cite{CäsSät} und Kalziumdampf \cite{KalSät} untersucht werden und bleibt damit auch weiterhin ein wichtiger Bestandteil für die Wissenschaft um atomare Übergänge explizit untersuchen und auflösen zu können. 

\cleardoublepage{}