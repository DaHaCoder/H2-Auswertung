\chapter{Einführung und Motivation}
\label{chap:introduction}

\epigraph{\textit{The test of all knowledge is experiment}. Experiment is the \textit{sole judge} of scientific ``truth''.}{\textit{Richard Feynman}, \cite[S. 1]{feynman1964}}

\noindent Die moderne Laserspektroskopie ist eine der exaktesten Methoden, um Konzepte und Modelle aus der theoretischen Physik zu überprüfen. \\
Heutzutage sind wir in der Lage, mittels dieser Methodik Aussagen über die kleinsten Bausteine der Natur, in einer Größenordnung von $10^{-18}\text{m}$ empirisch zu belegen. Dabei bedienen wir uns einem Beobachtungsphänomen, welches als erster Beleg für die quantisierte Natur der Materie galt: Die Absorptions- und Emissionslinien. 
\noindent Durch die detektierte Intensitätsverteilung, die auch als Spektrum bezeichnet wird, ist es möglich die eingegangene Strahlung mit hoher spektraler Auflösung auf ihre energetischen Eigenschaften zu untersuchen, um somit Rückschlüsse über die zu untersuchenden Atome und Moleküle ziehen zu können.\\
\noindent Die Verwendung eines Lasers als Strahlquelle ist für die Präzision in dieser Messung von zentraler Bedeutung, da wir uns dessen fundamentale Eigenschaften wie Kohärenz, Polarisation und Richtung in unserem Experiment zu Nutzen machen.\\
\noindent Die hier verwendete AlGaAs-Laserdiode, emittiert als Infrarotstrahler eine Wellenlänge von $\lambda = \SI{780}{\nano \meter}$ und findet neben ihrer spektroskopischen Anwendung auch Einsatz in Laserdruckern, in Lichtschranken oder in allgemein bekannten CD-Laufwerken. Mit anderen Worten ist es uns also gelungen mit einem Laser unseres alltäglichen Lebens Atome spektroskopisch zu untersuchen.   \\

\noindent In diesem Versuch soll die Genauigkeit der Laserspektroskopie anhand der Aufspaltung der quantisierten Energiespektren von Rubidium demonstriert werden. Deshalb werden im nächsten Kapitel zunächst die dafür essenziellen, grundlegenden physikalischen Konzepte erklärt, die experimentelle Durchführung skizziert um abschließend die Messergebnisse zu diskutieren und das Resultat -- die vollständige Auswertung der $D_{2}$-Linie von Rubidium -- vorzustellen.\\


\cleardoublepage{}