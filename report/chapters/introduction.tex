\chapter{Einführung und Motivation}
\label{chap:introduction}

\epigraph{\textit{The test of all knowledge is experiment}. Experiment is the \textit{sole judge} of scientific ``truth''.}{\textit{Richard Feynman}, \cite[S. 1]{feynman1964}}

\noindent Die moderne Laserspektroskopie ist eine der exaktesten Methoden, um Konzepte und Modelle aus der theoretischen Physik zu überprüfen. \\
Heutzutage sind wir in der Lage, mittels dieser Methodik Aussagen über kleinste Bausteine der Natur, in einer Größenordnung von $10^{-18}\text{m}$ empirisch zu belegen. Dabei bedienen wir uns eines Beobachtungsphänomens, welches als erster Beleg für die quantisierte Natur der Materie galt: Absorptions- und Emissionslinien. \\
\noindent Unter Verwendung der AlGaAs-Laserdiode, die als Infrarotstrahler eine Wellenlänge von $\lambda = \SI{780}{\nano \meter}$ emittiert, findet neben ihrer spektroskopischen Anwendung auch Einsatz in Laserdruckern, in Lichtschranken oder in allgemein bekannten CD-Laufwerken.\\

\noindent In diesem Versuch soll die Genauigkeit der Laserspektroskopie anhand der Aufspaltung der quantisierten Energiespektren von Rubidium demonstriert werden. Deshalb werden im nächsten Kapitel zunächst die dafür essenziellen, grundlegenden physikalischen Konzepte erklärt, die experimentelle Durchführung skizziert um abschließend die Messergebnisse zu diskutieren und das Resultat -- die vollständige Auswertung der $D_{2}$-Linie von Rubidium -- vorzustellen.\\


\cleardoublepage{}