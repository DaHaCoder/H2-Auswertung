%%%%%%%%%%%%%%%%%%%%
%%% ALL PACKAGES %%%
%%% ============ %%%

%%%%%%%%%%%%%%%%%%%%%%%% GENERAL PACKAGES %%%%%%%%%%%%%%%%%%%%%%%
\usepackage[utf8]{inputenc}                                         %   for accepting different input encodings [STANDARD PACKAGE] -- https://www.ctan.org/pkg/inputenc
\usepackage[T1]{fontenc}                                            %   for selecting font encodings [STANDARD PACKAGE] -- https://www.ctan.org/pkg/fontenc
\usepackage[ngerman]{babel}                                         %   for english language -- https://www.ctan.org/pkg/babel
\usepackage{datetime}                                               %   for date and time   -- https://www.ctan.org/pkg/datetime
\usepackage{lmodern}                                                %   for text font as Latin Modern -- https://www.namsu.de/Extra/pakete/Lmodern.html
\usepackage[left=2cm,right=2cm,top=2cm,bottom=3cm]{geometry}        %   for page/geometry layout -- https://www.ctan.org/pkg/geometry
\usepackage{fancyhdr}                                               %   for fancy template  -- https://www.ctan.org/pkg/fancyhdr
\usepackage{import}                                                 %   for import of other files, i.e. a .tex file with pgfplots  -- https://www.ctan.org/pkg/import
\usepackage[backend=biber,style=alphabetic,sorting=ynt]{biblatex}   %   for using bibliography, references and citations -- https://ctan.org/pkg/biblatex
\usepackage{comment}                                                %   for extra comment features  -- https://www.ctan.org/pkg/comment
\usepackage{float}                                                  %   for positioning figures and tables -- https://ctan.org/pkg/float
\usepackage[section]{placeins}                                      %   for controlling float placement -- https://ctan.org/pkg/placeins
\usepackage{epigraph}                                               %   for quotes at the beginning of a chapter -- https://ctan.org/pkg/epigraph
%%%%%%%%%%%%%%%%%%%%%%%%%%%%%%%%%%%%%%%%%%%%%%%%%%%%%%%%%%%%%%%%%%%%%

%%%%%%%%%%%%%%%%% MATH, SCIENCE AND SPECIAL SYMBOL PACKAGES %%%%%%%%%%%%%%%%%%%%%
\usepackage{amsmath, amsfonts, amssymb, amsthm, nccmath, bbm, mathdots}         %   for mathematics, i.e. bbm for identity matrix 
%   -- https://www.ctan.org/pkg/amsmath                                         %
%   -- https://www.ctan.org/pkg/amsfonts                                        %        
%   -- https://www.ctan.ebinger.cc/tex-archive/fonts/amsfonts/doc/amssymb.pdf   %                                                                            
%   -- https://www.ctan.org/pkg/amsthm                                          %
%   -- https://www.ctan.org/pkg/nccmath                                         %
%   -- https://www.ctan.org/pkg/bbm                                             %            
%   -- https://www.ctan.org/pkg/mathdots                                        %
\usepackage[nointegrals]{wasysym}                                               %   for more symbols, i.e. astronomical symbols                                   
\usepackage{physics}                                                            %   for physics, i.e. brakets
\usepackage{siunitx}                                                            %   for using si-units
\usepackage{fontawesome}                                                        %   for other nice symbols
                                                                                %   for mathematical enhancements in LaTeX (American Mathematical Society), have a look at https://www.ctan.org/pkg/amslatex
                                                                                %   for mathematical and scientific symbols, have a look a at 'The Comprehensive LaTeX Symbol List' -- https://ftp.mpi-inf.mpg.de/pub/tex/mirror/ftp.dante.de/pub/tex/info/symbols/comprehensive/symbols-a4.pdf 
%%%%%%%%%%%%%%%%%%%%%%%%%%%%%%%%%%%%%%%%%%%%%%%%%%%%%%%%%%%%%%%%%%%%%%%%%%%%%%%%%

%%% COLORS AND GRAPHICS PACKAGES %%%
\usepackage[table]{xcolor}         %   for using colors
\usepackage{empheq}                %   for using emphasizing equations
\usepackage[most]{tcolorbox}       %   for colored boxes
\usepackage{realboxes}             %   for more box options
\usepackage{graphicx}              %   for using graphics
\usepackage{eso-pic}               %   for picture options
\usepackage{transparent}           %   for transparency in pictures
\usepackage{standalone}            %   for compiling pictures and garphics in a seperate .tex-file and including it in the main document -- https://www.ctan.org/pkg/standalone
\usepackage{tikz}                  %   for creating beatiful and precise graphics, i.e. picture of spherical coordinates -- https://www.ctan.org/pkg/pgf
\usepackage{pgfplots}              %   for creating plots in two or three dimensions -- https://www.ctan.org/pkg/pgfplots
%%%%%%%%%%%%%%%%%%%%%%%%%%%%%%%%%%%%

%%% NUMERATE AND HIGHLIGHTING PACKAGES %%%
\usepackage{caption}                     %   for more caption options in graphics
\usepackage{enumerate}                   %   for numerating, i.e. sections 
\usepackage{hyperref}                    %   for highlighting/reference of links, i.e. websites
\usepackage{listings, minted}            %   for highlighting code
\usepackage{pythonhighlight}             %   for highlighting python code
%%%%%%%%%%%%%%%%%%%%%%%%%%%%%%%%%%%%%%%%%%


\usepackage{blindtext}
\newcommand{\bt}{\blindtext}


%%%%%%%%%%%%%%%%%%%%
%%% OWN COMMANDS %%%
%%% ============ %%%

%%% MATH COMMANDS %%%
% -- own commands for math symbols -- %
\newcommand{\N}{\mathbb{N}}           %   for the set of natural numbers
\newcommand{\Z}{\mathbb{Z}}           %   for the set of integers
\newcommand{\Q}{\mathbb{Q}}           %   for the set of rational numbers
\newcommand{\R}{\mathbb{R}}           %   for the set of real numbers
\newcommand{\C}{\mathbb{C}}           %   for the set of complex numbers
\newcommand{\D}{\mathrm{d}}           %   for mathroman d, i.e. differential d
\newcommand{\E}{\mathrm{e}}           %   for mathroman e, i.e. Euler's constant
\newcommand{\I}{\mathrm{i}}           %   for mathroman i, i.e. imaginary unit
\newcommand{\1}{\mathbbm{1}}          %   for identity matrix
\newcommand{\bs}{\boldsymbol}         %   for bold math symbols (instead of \textbf{} or \mathbf{}), i.e. for vectors in physcs 
\DeclareMathOperator{\diag}{diag}     %   for writing a diagonal matrix, i.e. sign convetion of Minkowski metric tensor \eta_{\mu \nu} = \diag(-,+,+,+)
\newcommand{\latex}{\LaTeX\xspace}    %   for the LaTeX symbol


%%% E-MAIL COMMAND %%%
% -- own command to insert e-mail -- %
\newcommand{\email}[1]{\href{mailto:#1}{#1}}
%%%%%%%%%%%%%%%%%%%%%

%%% COMMANDS FOR TITLEPAGE %%%
\newcommand{\getTitle}{Fortgeschrittenenpraktikum II: Laserspektroskopie (H2)}
\newcommand{\getSubtitle}{Auswertung}
\newcommand*{\getAuthorOne}{Danial Hagemann}
\newcommand*{\getAuthorTwo}{Tena Matosevic}
\newcommand*{\getLocation}{München}
%%%%%%%%%%%%%%%%%%%%%%%%%%%%%%%

%%% COMMAND FOR PRETTY CODE HIGHLIGHTING %%%
% -- have a look at https://tex.stackexchange.com/questions/140166/making-inline-code-printing-pretty?noredirect=1&lq=1 -- %
%\newcommand\code[3][]{
%    \tikz[baseline=(s.base)]{
%        \node(s)[
%             rounded corners,
%             fill=blue!5,        % background color
%             draw=gray,          % border of box
%             %text=gray!50!black, % text color
%             inner xsep =3pt,    % horizontal space between text and border
%             inner ysep =0pt,    % vertical space between text and border
%             text height=2ex,    % height of box
%             text depth =1ex,    % depth of box
%             #1                  % other options
%         ]{\mintinline{#2}{#3}};
%     }
% }
 
\definecolor{mygreen}{rgb}{0,0.6,0}
\definecolor{mygray}{rgb}{0.8,0.8,0.8}
\definecolor{mymauve}{rgb}{0.58,0,0.82}

\lstset{ %
  backgroundcolor=\color{mygray},       % choose the background color
  basicstyle=\ttfamily,                 % size of fonts used for the code
  breaklines=true,                      % automatic line breaking only at whitespace
  captionpos=b,                         % sets the caption-position to bottom
  commentstyle=\color{mygreen},         % comment style
  escapeinside={\%*}{*)},               % if you want to add LaTeX within your code
  keywordstyle=\color{blue},            % keyword style
  numbers=left,                         % alignment of numbers
  numberstyle={\small \color{black}},    % number style
  numbersep=9pt,                        % this defines how far the numbers are from the text
  stepnumber=1,
  stringstyle=\color{mymauve}           % string literal style
}
%%%%%%%%%%%%%%%%%%%%%%%%%%%%%%%%%%%%%%%%%%%


%%%%%%%%%%%%%%%%%%%%%%%
%%% PACKAGE SETUPS %%%%
%%% ============== %%%% 

%%% GEOMETRY SETUP %%%
 \geometry{
 a4paper,
 total={170mm,257mm},
 left=20mm,
 right=20mm,
 top=20mm,
 bottom=30mm
 }
%%%%%%%%%%%%%%%%%%%%%%

%%% FANCYHDR AND LAYOUT SETUP %%%%   
\pagestyle{fancy}
\fancyhf{}
\fancyhead[RE]{\normalfont\bfseries\sffamily\leftmark}
\fancyhead[LO]{\normalfont\bfseries\sffamily\rightmark}
\fancyhead[RO,LE]{\thepage}

\fancypagestyle{plain}{%
  \fancyhf{}
  \fancyhead[RO,LE]{\thepage}
}

\makeatletter
\renewcommand\chaptermark[1]{%
  \markboth{%
    \ifnum \c@secnumdepth >\m@ne
      \if@mainmatter
        %\chaptername
      \fi
    \fi
    #1%
  }{}
}

\renewcommand\sectionmark[1]{\markright{\thesection\enspace #1}}
%\renewcommand\subsectionmark[1]{\markright{\thesubsection\enspace #1}}
\makeatother 
\setcounter{secnumdepth}{\subsubsectionnumdepth}
%%%%%%%%%%%%%%%%%%%%%%

%%% SI SETUPS %%%
\sisetup{locale=DE}                     
\sisetup{per-mode = symbol-or-fraction} 
%%%%%%%%%%%%%%%%%

%%% TIKZ AND PGFPLOTS SETUP %%%
% \usetikzlibrary{intersections}
% \usetikzlibrary{decorations.pathreplacing, decorations.markings}
% \usetikzlibrary{tikzmark}
% \pgfplotsset{every axis/.style={scale only axis}, compat = newest}
%%%%%%%%%%%%%%%%%%%%%%%%%%%%%%%

%%% HYPER SETUP %%% 
\hypersetup{   
    colorlinks = true,     
    linkcolor = blue,
    citecolor = violet,
    filecolor = magenta,      
    urlcolor = blue       
}                    
%%%%%%%%%%%%%%%%%%%%%%