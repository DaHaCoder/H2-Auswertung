\begin{appendix}


\chapter{Anhang: Datensätze und Quellcodes}
\label{app:data-and-source-code}

Alle Datensätze und Quellcodes, welche dieser Auswertung zu Grunde liegen, sind auf \href{https://github.com/DaHaCoder/H2-Auswertung}{GitHub}\footnote{\href{https://github.com/DaHaCoder/H2-Auswertung}{https://github.com/DaHaCoder/H2-Auswertung}} zu finden. 
Im Folgenden wird aufgelistet, welche Datensätze und Quellcodes den Abbildungen und Tabellen zu Grunde liegen:
\begin{enumerate}[1.]
    \item Datensatz: \href{https://github.com/DaHaCoder/H2-Auswertung/blob/main/data/data00.csv}{\lstinline{data-00.csv}} 
    \begin{itemize}
        \item Abbildung \ref{fig:plot-data00-all} [Quellcode: \href{https://github.com/DaHaCoder/H2-Auswertung/blob/main/code/plot-data00-all.py}{\lstinline{plot-data00-all.py}}]
        \item Abbildung \ref{fig:plot-data00-resonator} [Quellcode: \href{https://github.com/DaHaCoder/H2-Auswertung/blob/main/code/plot-data00-resonator.py}{\lstinline{plot-data00-resonator.py}}]
        \item Abbildung \ref{fig:plot-data00-rubidium} [Quellcode: \href{https://github.com/DaHaCoder/H2-Auswertung/blob/main/code/plot-data00-rubidium.py}{\lstinline{plot-data00-rubidium.py}}]
        \item Abbildung \ref{fig:plot-data00-rubidium-normalized-fit} [Quellcode: \href{https://github.com/DaHaCoder/H2-Auswertung/blob/main/code/plot-data00-rubidium.py}{\lstinline{plot-data00-rubidium.py}}]
        \item Tabelle \ref{tab:plot-data00-rubidium} [Quellcode: \href{https://github.com/DaHaCoder/H2-Auswertung/blob/main/code/plot-data00-rubidium.py}{\lstinline{plot-data00-rubidium.py}}]
        \item Tabelle \ref{tab:delta_nu_parameter} [Quellcode: \href{https://github.com/DaHaCoder/H2-Auswertung/blob/main/code/plot-data00-rubidium.py}{\lstinline{plot-data00-rubidium.py}}]
    \end{itemize}
    \item Datensatz: \href{https://github.com/DaHaCoder/H2-Auswertung/blob/main/data/data20-gain30-01.csv}{\lstinline{data20-gain30-01.csv}}
    \begin{itemize}
        \item Abbildung \ref{fig:plot-data20-gain30-01-rubidium} [Quellcode: \href{https://github.com/DaHaCoder/H2-Auswertung/blob/main/code/plot-data20-gain30-01-rubidium.py}{\lstinline{plot-data20-gain30-01-rubidium.py}}]
        \item Abbildung \ref{fig:plot-data20-gain30-01-dip-1-rubidium-normalized-fit} [Quellcode: \href{https://github.com/DaHaCoder/H2-Auswertung/blob/main/code/plot-data20-gain30-01-rubidium.py}{\lstinline{plot-data20-gain30-01-rubidium.py}}]
        \item Abbildung \ref{fig:plot-data20-gain30-01-dip-2-rubidium-normalized-fit} [Quellcode: \href{https://github.com/DaHaCoder/H2-Auswertung/blob/main/code/plot-data20-gain30-01-rubidium.py}{\lstinline{plot-data20-gain30-01-rubidium.py}}]
        \item Tabelle \ref{tab:plot-data20-gain30-01-dip-1} [Quellcode: \href{https://github.com/DaHaCoder/H2-Auswertung/blob/main/code/plot-data20-gain30-01-rubidium.py}{\lstinline{plot-data20-gain30-01-rubidium.py}}]
        \item Tabelle \ref{tab:plot-data20-gain30-01-dip-2} [Quellcode: \href{https://github.com/DaHaCoder/H2-Auswertung/blob/main/code/plot-data20-gain30-01-rubidium.py}{\lstinline{plot-data20-gain30-01-rubidium.py}}]
        \item Tabelle \ref{tab:gamma-parameter-dip-1} [Quellcode: \href{https://github.com/DaHaCoder/H2-Auswertung/blob/main/code/plot-data20-gain30-01-rubidium.py}{\lstinline{plot-data20-gain30-01-rubidium.py}}]
        \item Tabelle \ref{tab:gamma-parameter-dip-2} [Quellcode: \href{https://github.com/DaHaCoder/H2-Auswertung/blob/main/code/plot-data20-gain30-01-rubidium.py}{\lstinline{plot-data20-gain30-01-rubidium.py}}]
    \end{itemize}
    \item Datensatz: \href{https://github.com/DaHaCoder/H2-Auswertung/blob/main/data/data20-gain30-03.csv}{\lstinline{data20-gain30-03.csv}}
    \begin{itemize}
        \item Abbildung \ref{fig:plot-data20-gain30-03-rubidium} [Quellcode: \href{https://github.com/DaHaCoder/H2-Auswertung/blob/main/code/plot-data20-gain30-03-rubidium.py}{\lstinline{plot-data20-gain30-03-rubidium.py}}]
        \item Abbildung \ref{fig:plot-data20-gain30-03-rubidium-normalized-fit} [Quellcode: \href{https://github.com/DaHaCoder/H2-Auswertung/blob/main/code/plot-data20-gain30-03-rubidium.py}{\lstinline{plot-data20-gain30-03-rubidium.py}}]
        \item Tabelle \ref{tab:plot-data20-gain30-03} [Quellcode: \href{https://github.com/DaHaCoder/H2-Auswertung/blob/main/code/plot-data20-gain30-03-rubidium.py}{\lstinline{plot-data20-gain30-03-rubidium.py}}]
    \end{itemize}
    \item Datensatz: \href{https://github.com/DaHaCoder/H2-Auswertung/blob/main/data/data20-gain30-04.csv}{\lstinline{data20-gain30-04.csv}}
    \begin{itemize}
        \item Abbildung \ref{fig:plot-data20-gain30-04-rubidium} [Quellcode: \href{https://github.com/DaHaCoder/H2-Auswertung/blob/main/code/plot-data20-gain30-04-rubidium.py}{\lstinline{plot-data20-gain30-04-rubidium.py}}]
        \item Abbildung \ref{fig:plot-data20-gain30-04-rubidium-normalized-fit} [Quellcode: \href{https://github.com/DaHaCoder/H2-Auswertung/blob/main/code/plot-data20-gain30-04-rubidium.py}{\lstinline{plot-data20-gain30-04-rubidium.py}}]
        \item Tabelle \ref{tab:plot-data20-gain30-04} [Quellcode: \href{https://github.com/DaHaCoder/H2-Auswertung/blob/main/code/plot-data20-gain30-04-rubidium.py}{\lstinline{plot-data20-gain30-04-rubidium.py}}]
    \end{itemize}
\end{enumerate}

\cleardoublepage{}

\chapter{Anhang: Zeitabstände der Resonator-Peaks}
\label{app:time-to-freq}

\begin{table}[!h]
    \centering
    \begin{tabular}{|c|c|}
        \hline
         peaks  $\#i$, $\#i+1$    &   $t_{i, i+1}$ in \SI{}{\milli \second} \\
         \hline
         \hline
         \#1, \#2    &   7,4416         \\
         \hline
         \#2, \#3    &   6,8432         \\
         \hline
         \#3, \#4    &   6,2992         \\
         \hline
         \#4, \#5    &   6,1408         \\
         \hline
         \#5, \#6    &   5,9376         \\
         \hline
         \#6, \#7    &   5,7712         \\
         \hline
         \#7, \#8    &   5,5952         \\
         \hline
         \#8, \#9    &   5,5664         \\
         \hline
         \#9, \#10   &   5,3872         \\
         \hline
         \#10, \#11  &   5,3216         \\
         \hline
         \#11, \#12  &   5,2272         \\
         \hline
         \#12, \#13  &   5,1312         \\
         \hline
         \#13, \#14  &   5,0912         \\
         \hline
         \#14, \#15  &   5,0320         \\
         \hline
         \#15, \#16  &   5,0336         \\
         \hline
    \end{tabular}
    \caption{Zeitabstände benachbarter Resonator-Peaks von links nach rechts}
    \label{tab:time-distances}
\end{table}

\cleardoublepage{}


\chapter{Anhang: Temperaturbestimmung der Rubidiumzelle}
\label{app:temperature}

Mittels der Formel \eqref{eq:delta-omega} lässt sich die Temperature der Rubidiumzelle wie folgt bestimmen:
\begin{align*}
    \Delta \nu &= 2 \sqrt{\ln(2)} \frac{\nu_{0}}{c} \sqrt{\frac{2 k_{B} T}{m}} = \frac{\nu_{0}}{c} \sqrt{\frac{8 \ln(2) k_{B} T}{m}} \\
    \Rightarrow T &= \frac{m c^2}{8 \ln(2) k_{B}} \biggl(\frac{\Delta \nu}{\nu_{0}} \biggr)^2.
\end{align*}

\noindent Zunächst berechnen wir, welches Verhältnis von Halbwertsbreite $\Delta \nu$ zur Peak Position $\nu_{0}$ bei Raumtemperatur ($T \approx \SI{300}{\kelvin}$) für $^{85}$Rb ($m =: m_{85} \approx \SI{1,409e-25}{\kilogram}$, siehe \cite{Steck85}, S. 16, Tabelle 2) und für $^{87}$Rb ($m =: m_{87} \approx \SI{1,443e-25}{\kilogram}$, siehe \cite{Steck87}, S. 17, Tabelle 2) zu erwarten ist.
\begin{align*}
    \biggl(\frac{\Delta \nu}{\nu_{0}}\biggr)_{85} & \approx \sqrt{\frac{8 \ln(2) \cdot  \SI{1,38e-23}{\frac{\joule}{\kelvin} \cdot \SI{300}{\kelvin}}}{\SI{1,409e-25}{\kilogram} \cdot (\SI{299792458}{\frac{\meter}{\second}})^2}} \approx \SI{1,346e-6}{}\\ \\
    \biggl(\frac{\Delta \nu}{\nu_{0}}\biggr)_{87} & \approx \sqrt{\frac{8 \ln(2) \cdot  \SI{1,38e-23}{\frac{\joule}{\kelvin} \cdot \SI{300}{\kelvin}}}{\SI{1,443e-25}{\kilogram} \cdot (\SI{299792458}{\frac{\meter}{\second}})^2}} \approx \SI{1,330e-6}{}
\end{align*}

\end{appendix}
